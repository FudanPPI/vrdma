\section{Discussion}
In this section, there are some concerns about uniRDMA in cloud environments:
\begin{itemize}
	\item {\verb|Control police for clouds|}: To manage the resources precisely, there are lots of important polices in clouds, e.g. QoS, metering and so on. In uniRDMA, even though all RDMA commands in data path are executed in applications' user space, we can still meet the control features in virtual layer for these resons:  First,  the virtual layer can monitoring the QPs or CQs for RDMA traffic information, such as,  size and status of one RDMA work request, because all RDMA resources (including QPs and CQs) are mapped at host's virtual layer; Second, the virtual layer can  control the virtual RDMA traffic by controling the RDMA resources,  with the help of host RDMA libraries. For an example,  we can destroy the QP in the virtual layer when a RDMA connections' traffic is excessive.  These will cause more CPU overhead in the host virtual layer, but would not impact the performance of RDMA applications in clouds.

	\item {\verb|Virtual Instances Migration|}: Migration is important  for both containers and VMs in clouds with many benifits, e.g. resource utilization and fail-over. With the virtual RDMA network , uniRDMA can support offline migration without reconfiguring the physical RDMA network for applications. In specific,after rebooting the migrated virtual instance, the application can rebuild the RDMA connection  only by modifing the address mapping in software virtual layer. Currently, for live migrations, it is still hard because memory regions in RDMA application may be uncertain under bypassing or one-side communication. Thus, uniRDMA does not support live migrations for both VMs and containers.
	
	\item {\verb|Other network extensions|}: RDMA can also be exploited to optimize the performance of other network applications,  such as TCP/IP.  Existing works include vSocket ~\cite{wang2019vsocket}, SockDirect~\cite{li2019socksdirect} so on. In uniRDMA, we can extend the simiar design for socket applications in containers and VMs: unified vNICs and the driver of vNICs in containers or VMs.  The difference in the virtual layer is mainly how to emulate the vNICs based on host RDMA's libraries. 
	
\end{itemize}
