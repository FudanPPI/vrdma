\section{Introduction}
RDMA (remote direct memory access) is a high-performance network,  with high throughput, low latency and low CPU load. Thus, RDMA is widely adopted in supercomputing for various applications,  such as artificial intelligence, and big data processing. 

Recently, migrating to cloud is a trend for supercomputing users. Different from traditional HPC environment, resource provisioning are done in the form of virtual machines or containers in clouds~\cite{hpc-cloud}. Thus, to use RDMA for applications in clouds, RDMA devices need to be virtualzed. Existing solutions for RDMA virtualization include software-based and hardware-based solutions. The software-based solutions are mainly about HyV, FreeFlow or MasQ~\cite{pfefferle2015hybrid}~\cite{kim2019freeflow}~\cite{he2020masq}. FreeFlow is  designed for containers and HyV/MasQ are proposed for virtual machines. The hardware-based solution is based on PCI pass-through for less overhead, like SR-IOV~\cite{sr-iov}. 

However, in practice, hybrid virtual environments are common for clouds. In specific, VMs, containers orother form of virtualization (e.g. containers in VMs) can be found in the same datacenter or even on the same host machine. The existing solutions are not suitable for hybrid virtual environments for the following reasons:

For software-based solutions, they are designed for specific virtual environments. Thus, in hybrid virtual environments, we should deploy multiple frameworks or extend one framework for another environment. If multiple frameworks co-exist in the same cluster, RDMA resources should be divided statically to avoid management conflict. This may cause the low resource utilization and higher management complexity. If single framework is extended to hybrid virtual environments: for FreeFlow, the communication between applications and the virtual layer does not suit VMs and it has apperant overhead in RDMA network without bypassing the virtual layer; for HyV or MasQ, compared to FreeFlow, their virtual layers are in kernel-space that brings new problems: the kernel's attack surface is larger due to lots of inserted code in kernel, managment development is inflexible in kernel programming, and the inserted modules are depenedent on hardware-specific RDMA kernel drivers.

For hardware-based solutions (e.g. SR-IOV), RDMA device resources are directly allocated to virtual machines or containers. Thus, the device utilization is static and inefficient. Moreover, RDMA networks in clouds are still managed by physical switches or routers. Thus, virtual network management is not scalable and portable in large-scale clouds.

To address these problems, we propose a unified RDMA virtualization framework for hybrid virtual environment in clouds, namely uniRDMA, also with the design goals of flexible management and high-performance. Based on uniRDMA, applications in VMs or containers can diretly use the vRNICs (virtual RDMA network interface cards) for RDMA. vRNICs are unified device encapsulation in host user-space and managed under the virtual layer in each host server.

As the basic abstract unit of uniRDMA, vRNIC is virtualized and provided to RDMA applications. Each vRNIC is virtualized with complete attributes like physical RNIC, such as QP (Queue Pair) and DB (DoorBell registers) and all vRNICs are in userspace for flexible manageability and hiding kernel driver’s details. To utilize the high-performance of physical RNIC, vRNICs are mapped to the VF (the hardware-based virtual function) in RNIC respectively.

The driver of vRNIC is for RDMA applications in virtual environments. Note that VMs and containers are different visualizations in one server. Thus, the communication between driver and vRNIC devices are obviously different. To address it, the same communication protocols are proposed from applications to containers or VMs’ hypervisors. And a specific driver in guest OS is designed to transport applications’ commands in VMs. Moreover, for high-performance, we map the virtual RDMA resources between the vRNIC and application to achieve zero-copy and by-pass like native RDMA.

Virtual layer is responsible for RDMA devices management and virtual RDMA network. For RDMA device management,  under the limited VFs, we make dynamic vRNIC mapping in the virtual layer to get higher scalability. Besides, the virtual layer is software virtual switch for virtual RDMA network management, such as vRNIC address, routing rules.

Finally, we implement the prototype and evaluate it in different aspectsbenchmarks, such as throughput, latency, scalability, with and real-word applications. From the result, uniRDMA’s performance is close to native RDMA in hybrid virtual environment and the overhead is less than 5\% to hardware-based virtualization. uniRDMA also has high scalability and adapts to real-world RDMA applications in hybrid cloud environments. The main contributions in this paper are as follows:

\begin{itemize}
\item Unified RDMA virtualization in hybrid virtual environments is firstly proposed in this paper, and it is stable for different kernels or RDMA devices. Besides, high performance and high manageability are maintained.  

\item By utilizing vhost-user and existing system calls, we realize the mapping of RDMA resources in userspace to achieve high-performance.  

\item Unified framework is evaluated and the results proved that uniRDMA maintains high performance close to native RDMA.
\end{itemize}

