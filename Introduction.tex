\section{Introduction}
 Supercomputing has been widely adopted in different fields for high-performance, such as artificial intelligence, and big data processing. Recently, most supercomputing is going cloud for effiecncy and elasticity, especially in hybird virtual environment, which includes both virtual machines and containers, such as AWS HPC Cloud~\cite{aws-hpc}. 

As the popular network in supercomputing, RDMA has high throughput, low latency and low CPU load. To expolit RDMA's high-performance in hybrid virtual environment, RDMA virtualization is necassery. Howerver, neither of existing solutions suits for hybrid virtual environment due to lack of network performance or unified management.

Exsiting solutions are mainly hardware-based or software-based. The representative hardware-based virtualization is SR-IOV~\cite{sr-iov}. SR-IOV lacks scabality and portability without software virtual layer. In software virtualization, existing solutions are not adopted to hybrid virtual environment. FreeFlow is only designed for containers and ineffective due to the overhead of data commands forwarding.  HyV~\cite{pfefferle2015hybrid} ~\cite{pfefferle2014vverbs} and MasQ~\cite{he2020masq} are proposed for virtual machines.

To address these problems, we proposes an unified RDMA virtualization framework for hybrid virtual environment, which achieves high performance and high manageability in hybrid virtual environment. It is mainly consisted of single centralized virtual layer and driver for vRNICs in virtual layer. All managements are concentrated in the user space virtual layer, such as isolation and virtual network management; the driver is general in hybrid virtual environment. 

% 虚拟层连接物理网卡和虚拟实例,挑战在于在利用底层网卡高性能特点的同时实现高管理 
In unified virtual layer, which virtualize RNIC(physical RDMA network card) to vRNICs(virtual RNIC), the main challenge is to make vRNICs both manageable and high-performance. So, vRNICs(virtual RDMA network cards) are virtualized in user space and bonded to isolated VFs with the help of hardwared-based virtual solution SR-IOV. For virtual network management, vRNIC is configured with virtual address and the address mapping tables are maintained in virtual layer. To achive high-performance meanwhile, each vRNIC is dynamicly mapped into the VF to utilize the hardware processor in RNIC.

For vRNIC driver, which connects the RDMA applications in virtual instances and the virtual layer's vRNICs, the main challenge is to make driver general and performance meanwhile. We found that both virtual machines and container applications are processes of host. So, we design the general driver for containers and VMs. Moreover, we map all RDMA resources in vRNIC for application to meet zero-copy and by-pass. 

Finnaly, we implement the prototype and evaluate it in different benchmarks, such as throughput, latency, scalability, and real-word applications. From the result, uniRDMA's performance is close to native RDMA in hybrid virtual environment and the overhead is less than 5\% to hardware-based virtualization. uniRDMA also has high scalability and adapts to real-world RDMA applications in hybrid cloud environments.The main contributions in this paper are as follows:

\begin{itemize}
\item Unified RDMA virtulization in hybrid virtual environments is firstly proposed in this paper and uniRDMA is general RDMA virtualization framework,  while maintaining high performance and high manageability.

\item uniRDMA are evaluated and the results proved that uniRDMA maintains high performance close to native RDMA.
\end{itemize}
