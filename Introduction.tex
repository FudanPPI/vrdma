\section{Introduction}
 Supercomputing has been widely adopted in different fields for high-performance, such as artificial intelligence, and big data processing. Recently, most supercomputing is in cloud for effiecncy and elasticity, especially in hybird virtual environments, which include both virtual machines and containers. For example, AWS HPC Cloud~\cite{aws-hpc}.

As the popular network in supercomputing, RDMA has high throughput, low latency and low CPU load. To expolit RDMA's high-performance in hybrid virtual environments, RDMA virtualization is necassery. Howerver, neither of existing solutions suit for hybrid virtual environments due to lack of network performance or unified management.

Exsiting solutions mainly about hardware-based and software-based. The representative hardware virtualization is SR-IOV~\cite{sr-iov}.  Although the isolation and high performance are maintained,  SR-IOV lacks portability and other manageability without software virtual layer. In software virtualization, existing solutions treat virtual machines and containers differently. For containers, FreeFlowl~\cite{kim2019freeflow} forwards all RDMA commands to the virtual layer, and that is ineffective. For virtual machines, HyV~\cite{pfefferle2015hybrid} ~\cite{pfefferle2014vverbs} avoids forwarding data commands by mapping RDMA resources, but lacks the management of virtual RDMA networks; although MasQ~\cite{he2020masq} makes up for this problem, but its virtual layer is in hypervisor and that is not suit for containers' user-space lightweight  management.

We proposes an unified RDMA virtualization framework for both containers and virtual machines, namely uniRDMA, which achieves high performance and high manageability in hybrid virtual environment. UniRDMA is mainly composed of single centralized uniRDMA virtual layer and general uniVerbs interfaces. All managements are concentrated in the user space virtual layer; the UniVerbs interface is general to the RDMA applications in virtual machines or containers.

We implements the prototype of uniRDMA and evaluate uniRDMA against with hardware virtualization, software virtualization FreeFlow, and native RDMA in different benchmarks, such as throughput, latency, scalability, and real-word applications. From the result, uniRDMA's performance is close to native RDMA in both virtual machine and container environments, and is significantly better than FreeFlow. The throughput can reach up to 6 times that of FreeFlow, and the latency can reach down to 40\% of FreeFlow.uniRDMA also has high scalability and adapts to real-world RDMA applications in hybrid cloud environments.The main contributions in this paper are as follows:

\begin{itemize}
\item Unified RDMA virtulization in hybrid virtual environments is firstly proposed in this paper and uniRDMA is general RDMA virtualization framework,  while maintaining high performance and high manageability.

\item uniRDMA are evaluated and the results proved that uniRDMA maintains high performance close to native RDMA.
\end{itemize}
