\section{Introduction}
Supercomputing has been widely adopted in different fields for high-performance, such as artificial intelligence, and big data processing. Recently, most supercomputing is going cloud for effiecncy and elasticity. In reality, cloud datacenters is becoming hybrid virtual environments, which includes both virtual machines, containers and other virtualization. Besides, there are different server environments in a datacenter, such as host OS, network devices.

For hybrid virtual environments, if lack of a unified virtualization, there are two disadvatanges: First, virtualization management are complicated beacuause of unintergated platform for different virtual instances; Second, resource utilization is low wihout of schedule collaboration. As the popular network in supercomputing, RDMA has high throughput, low latency and low CPU load. So, to utilize RDMA's high-performance in hybrid virtual environment, the first goal of RDMA virtualization is to be unified for containers, VMs and other virtualization. Besides, a stable unified framework for different host kernels or RDMA devices are also important. Howerver, neither of existing solutions are not unified for hybrid virtual environment.

Exsiting solutions are mainly hardware-based or software-based. The representative hardware-based virtualization is SR-IOV~\cite{sr-iov}. SR-IOV lacks enough management of RDMA because of its implementation in hardware. In software virtualization, existing solutions are not adopted to hybrid virtual environment. FreeFlow is only designed for containers and cannot be extended to VMs.  HyV~\cite{pfefferle2015hybrid} ~\cite{pfefferle2014vverbs} and MasQ~\cite{he2020masq} are proposed for VMs and they are dependent on specific RDMA device kernel driver and not portable for different host kernels or RDMA devices.

To address these problems, we proposes an unified RDMA virtualization framework for hybrid virtual environment, which is adopted to hybrid virtual environment with high performance and high manageability. Besides, it is stable for different kernel versions and RDMA devices. This framework is mainly consisted of multiple vRNICs which are managed in single virtual layer. Each vRNIC locates in userspace for flexible unified management and its driver is unified for containers and VMs.

% vRNIC 
As the basic unit of our virtual RDMA, vRNIC is virtualized and provided to RDMA applications. Each vRNIC is virtualized with complete arttributes like physical RNIC, such as QP(Queue Pair) and DB(DoorBell registers) and all vRNICs are in userspace for flexible manageability and hidding kernel driver's details. To utilize the high-performance of physical RNIC, vRNICs are mapped to the VF(the hardware-based virtual function) in RNIC respectively. 

% vRNIC driver
The driver of vRNIC is unified so that applications can use vRNICs transparently in both containers and VMs. To achieve it, we add a specific kernel driver in guest OS and use the same communication protocal as containers. However, we found that the performance is a critial problem even though the vRNIC is mapped to VF. So, we map the virtual RDMA resources between the vRNIC and application. After that, the zero-copy and by-pass are realized like native RDMA.

% virtual layer
To manage all vRNICs centralizly, we design a virtual layer in userspace. It is responsible for vRNIC instance virtualization and vRNIC's mapping. Especially, under the limited VFs, we makes dynamic vRNIC mapping management in virtual layer to get higher scalability. Besides, the virtual layer is software virtual switch for virtual RDMA network management, such as vRNIC address, routing rules.

Finnaly, we implement the prototype and evaluate it in different benchmarks, such as throughput, latency, scalability, and real-word applications. From the result, uniRDMA's performance is close to native RDMA in hybrid virtual environment and the overhead is less than 5\% to hardware-based virtualization. uniRDMA also has high scalability and adapts to real-world RDMA applications in hybrid cloud environments.The main contributions in this paper are as follows:

\begin{itemize}
\item Unified RDMA virtulization in hybrid virtual environments is firstly proposed in this paper, and it is stable for different kernels or RDMA device. Besides, high performance and high manageability are maintained.

\item Unified framework are evaluated and the results proved that uniRDMA maintains high performance close to native RDMA.
\end{itemize}
