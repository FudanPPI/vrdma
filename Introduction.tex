\section{Introduction}
RDMA(remote direct memory access), is a new type of high-performance network technology. RDMA is currently widely used in artificial intelligence, data processing, and high-performance computing. For examples, TensorFlow, Spark, and Hadoop [21] all have supported RDMA. With hardware protocol stack and zero copy technologies, RNICs(physical RDMA network cards) can bypass the kernel to read/write remote memory data according to the work requests of applications, without the participation of remote CPU. Therefore, RDMA has high throughput, low latency and low CPU load.

The core technology of cloud computing is virtualization, mainly including container and virtual machine. The container is a lightweight isolated runtime environment, does not need device emulation, and has low performance loss. The virtual machine has strong isolation and is more secure, but the performance loss is large. Both virtual machines and containers have been widely used, and the trend has become the unified deployment and management for hybrid virtual environments. For example, VMware's virtualization platform vSphere and ReadHat's container cloud platform OpenShift, both clearly support the unified deployment and management of virtual machines and containers in the latest version.

RDMA virtualization is necessary for cloud applications to exploit RDMA. RDMA virtualization not only needs to maintain high performance and manageability, but also have generality for hybrid virtual environments. Therefore, our RDMA virtualization goals are as follows:

\begin{itemize}
\item {\verb|Generality|}: To form unified RDMA virtualization, single centralized virtual layer should be set up, which is provided to virtual machines and containers with general interfaces.
\item {\verb|High performance|}: Virtual RDMA should be close to native RDMA in terms of throughput, latency, and CPU load. Meanwhile it should be suit for large-scale virtual cluster.
\item {\verb|High manageability|}: In RDMA virtualization, container and virtual machine characteristics should be maintained to realize portability, isolation and network management.
\end{itemize}

RDMA has different hardware characteristics and working mechanisms compared to traditional network. Therefore, RDMA virtualization is different from traditional network virtualization. Current RDMA virtualization work mainly includes hardware virtualization and software virtualization. However, none of the existing solutions can meet all above goals.

The representative of hardware virtualization is SR-IOV. Its virtual layer is located in the hardware. Although the isolation and high performance are maintained,  SR-IOV lacks portability and other manageability without software virtual layer. In software virtualization, existing solutions treat virtual machines and containers differently. For containers, FreeFlowl~\cite{kim2019freeflow} forwards all RDMA commands to the virtual layer, and that is ineffective because of losing RDMA's kernel by-pass. For virtual machines, HyV avoids forwarding overhead by mapping RDMA resources, but lacks the management of virtual RDMA networks; although MasQ makes up for this problem, its virtual layer is located in kernel space. Extending MasQ to the container environment will lose lightweight  management in user space for containers.

We proposes an unified RDMA virtualization framework for both containers and virtual machines, namely uniRDMA, which achieves high performance and high manageability. UniRDMA is mainly composed of single centralized uniRDMA virtual layer and general uniVerbs interfaces. All managements are concentrated in the user space virtual layer; the UniVerbs interface is general to the RDMA applications in virtual machines or containers.

There are mainly two challenges in the design of uniRDMA: First, virtual machines and containers are essentially different virtualization technologies. Virtual machines are under the management of hypervisor in kernel space, containers are isolted runtime mainly about user space. It is a challenge to build a centralized virtual layer for both containers and virtual machines; second, mapping all RDMA resources is the key idea to achive high performance. However, the existing solutions only implement the mapping operation in the same process for virtual machine. In this paper, the virtual layer belongs to another host process. So, it is also another challenge to map all RDMA resources.

To address the first challenge, uniRDMA separates the control and use of RDMA resources in virtulization, builds multiple vRNICs in single unified user space virtualization layer. Each vRNIC encapsulates virtual RDMA resources, and isolates with others by hardware virtualization. Moreover, through a common file-based shared memory queue, each vRNIC has a general interface for virtual machines and containers. For the second challenge,  at first, virtual RDMA resources in vRNICs are mapped to physical RNICs, and then shared memory are used to ensure RDMA resources in applications are mapped to vRNICs in the virtual layer.

We implements the prototype of uniRDMA and evaluate uniRDMA against with hardware virtualization, software virtualization FreeFlow, and native RDMA in different benchmarks, such as throughput, latency, scalability, and real-word applications. From the result, uniRDMA's performance is close to native RDMA in both virtual machine and container environments, and is significantly better than FreeFlow. The throughput can reach up to 6 times that of FreeFlow, and the latency can reach down to 40\% of FreeFlow.uniRDMA also has high scalability and adapts to real-world RDMA applications in hybrid cloud environments.

The main contributions in this paper are as follows:

(1)Unified RDMA virtulization in hybrid virtual environments is firstly proposed in this paper and uniRDMA is general RDMA virtualization framework,  while maintaining high performance and high manageability.

(2)uniRDMA are evaluated and the results proved that uniRDMA maintains high performance close to native RDMA.

