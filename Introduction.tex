\section{Introduction}
 Supercomputing has been widely adopted in different fields for high-performance, such as artificial intelligence, and big data processing. Recently, most supercomputing is going cloud for effiecncy and elasticity, especially in hybird virtual environment, which includes both virtual machines and containers, such as AWS HPC Cloud~\cite{aws-hpc}. 

As the popular network in supercomputing, RDMA has high throughput, low latency and low CPU load. To expolit RDMA's high-performance in hybrid virtual environment, RDMA virtualization is necassery. Howerver, neither of existing solutions suits for hybrid virtual environment due to lack of network performance or unified management.

Exsiting solutions are mainly hardware-based or software-based. The representative hardware-based virtualization is SR-IOV~\cite{sr-iov}. SR-IOV lacks scabality and portability without software virtual layer. In software virtualization, existing solutions are not adopted to hybrid virtual environment. FreeFlow is only designed for containers and ineffective due to the overhead of data commands forwarding.  HyV~\cite{pfefferle2015hybrid} ~\cite{pfefferle2014vverbs} and MasQ~\cite{he2020masq} are proposed for virtual machines.

To address these problems, we proposes an unified RDMA virtualization framework for hybrid virtual environment, namely uniRDMA, which achieves high performance and high manageability in hybrid virtual environment. uniRDMA is mainly consisted of single centralized uniRDMA virtual layer and general uniVerbs interfaces. All managements are concentrated in the user space virtual layer, such as isolation and virtual network management; the uniVerbs interfaces are general to RDMA applications in hybrid virtual environment. 

In unified virtual layer, the main challenges is to make isolated and managable for virtual RDMA. So, mutiple vRNICs(virtual RDMA network cards) are virtualized in user space. These vRNICs are isolated with the help of hardwared-based virtual solution SR-IOV. To utilize the high-performance of RNIC, each vRNIC is mappped into a VF in RNIC. For virtual network management, vRNIC is configured with virtual address and the address mapping tables are maintained in virtual layer.

For vRNIC interfaces, the main challenge is how to maintain generality and performance meanwhile. We found that both virtual machines and container applications are processes of host. So, we utilize shared memory as the general I/O interface between vRNICs and RDMA application in hybrid virtual environment. Moreover, shared memory is utilized to map RDMA resources in vRNIC to realize zero-copy. 

Finnaly, we implement the prototype of uniRDMA and evaluate uniRDMA in different benchmarks, such as throughput, latency, scalability, and real-word applications. From the result, uniRDMA's performance is close to native RDMA in hybrid virtual environment. uniRDMA also has high scalability and adapts to real-world RDMA applications in hybrid cloud environments.The main contributions in this paper are as follows:

\begin{itemize}
\item Unified RDMA virtulization in hybrid virtual environments is firstly proposed in this paper and uniRDMA is general RDMA virtualization framework,  while maintaining high performance and high manageability.

\item uniRDMA are evaluated and the results proved that uniRDMA maintains high performance close to native RDMA.
\end{itemize}
