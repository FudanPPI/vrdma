\section{Conclusion} \label{conclusion}

In this paper, we design a unified RDMA virtualization stack, namely \sys, which provides virtualized RDMA network for both VMs and containers on the same framework.
Through single point of management, RDMA resources could be flexibly allocated to VMs and containers. %With the support of \sys, one server machine could choose to host both virtualization form.
The experimental results show that the performance of \sys is similar to those of SR-IOV and \native for real applications.

Note: The authors will open source the \sys after the anonymous review period.

%In this paper, we design a unified RDMA virtualization stack, namely \sys, which provides virtualized RDMA network for both VMs and containers on the same framework.
%Through single point of management, RDMA resources could be flexibly allocated to VMs and containers. With the support of \sys, one server machine could choose to host both virtualization form.
%In \sys, the vRNIC device model are implemented to support both VMs and containers. The \sys core manages the vRNIC and physical RDMA resources and the management node manages the control and data plane policies across the RDMA network. The experimental results show that the performance overhead of \sys is less than 4\% compared to SR-IOV and about 10\% compared to \native.