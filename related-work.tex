\section{Related Work} \label{relatedwork}

\textbf{Traditional TCP/IP:}\quad TCP/IP is the basic network for each tenant in data centers. Several software solutions were proposed for traditional network virtualization. For both VMs and containers, virtual network interface cards (vRNICs) are one solution, e.g. virtio-net~\cite{virtio-russell2008}, vhost-net~\cite{vhost-net},  vhost-user-net~\cite{vhost-user-net} for VMs and veth~\cite{veth} for containers. These vRNICs are implemented in VM's or container's different space (i.e. kernel space and user space). Software-based switch, also knowed as virtual switch, in the local host brides the vNIC to physical NIC. Virtual switch can run in user space (e.g. Snabb Switch~\cite{snabb}), kernel space(e.g. Linux bridge~\cite{linux-bridge}), or both(e.g. Open vSwitch~\cite{ovs-2015}). 

\textbf{Hardware-based RDMA virtualization:}\quad SR-IOV~\cite{sr-iov} splits a physical PCI-e device to multiple PCI-e devices, which can be allocated to VMs, providing near-native performance. In comparison to software-baed virtualization, it lacks of refficient resource management and suffers from the flexibility issue due to its virtual layer is realized in hardware. In uniRDMA, the isolation of SR-IOV are utilized and the unscalable problems are solved by dynamic vRNIC mapping. 

\textbf{Software-based RDMA virtualization:}\quad FreeFlow~\cite{kim2019freeflow} forwards all RDMA commands from containers to the FreeFlow router (FFR) including data path. It intercepts the commands between applications and NIC drivers such as creating QP/CQ and sending/receiving data. The key different between FreeFlow and uniRDMA is that FreeFlow does capture data path commands to fitting in the container environment. Its data path commands, which accounts for the majority compared to the control path, costs extra forward latency. HyV~\cite{pfefferle2015hybrid} and MasQ~\cite{he2020masq} are designed for VMs. They achieve zero-copy and by-pass by mapping all resources. However, the implementation of memory mapping are different that is uniRDMA maps memory between virtual layer and VM's application. Moreover, HyV/MasQ backend is in kernel space while uniRDMA backend is in user space for more manageability and lightweight.