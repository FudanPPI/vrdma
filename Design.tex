% 第4章 设计
\section{Design}

% URN的目标是实现一个统一的软件虚拟化架构,能够同时为虚拟机和容器提供RDMA服务,并实现对虚拟RDMA的统一管理。此外,该框架还需具备与原生RDMA接近的高性能。在本章,介绍了URN中的一些关键性设计工作。
The goals of URN is to achieve a unified RDMA virtualization framework for both containers and VMs. In this framework, virtual RDMA network are managed centrally. Besides, performance are also should be made close to native RDMA. In this section, we introduce some key designs for above goals.


\subsection{vRNIC Design}

% vRNIC是一个软件的虚拟RDMA设备,包含前后端,前端负责为上层verb库提供接口并转发RDMA 命令,后端在主机端,通过与URN core或verbs库交互,以模拟guest对vRNIC的RDMA操作。 vRNIC继承了控制和数据通道分离的RDMA虚拟化工作,例如hyv和MasQ,控制路径上guest应用在本地的虚假RDMA上下文中执行verbs命令,并经过FE转发到BE维持的真实RDMA上下文中执行,之后将结果返回给guest;在数据路径上,将guest应用的RDMA资源与vRNIC BE上下文中的RDMA资源进行映射,guest应用的RDMA资源被注册到了物理网卡中,之后guest应用可以在本地直接使用RDMA资源进行数据操作。 在本文的工作中,后端位于主机用户空间,同时,整个架构扩展到了容器场景下。
vRNIC is a virtual device, which consists of frontend and backend. The frontend forward the commands of Verbs library to backend, and the backend execute the commands on physical RNIC through Verbs library in host user space. We inherited the existing solution HyV and extended it to hybrid virtual environments. Generally, the control path and data path are split when virtualization. Similar as HyV, In control path, the process is same as the above description, the control commands of applications in the guest, are forwarded to backend through the frontend and Verbs library. Besides, the RDMA resources in the guest are mapped into backend. Thus, in data path, upper applications and Verbs directly use the mapped RDMA resources from backends. The details of mapping is described in section Implexxx.

% 前端:FE-V 在 kernel space 原因, FE-C在user space原因,如何与Verbs交互;
% 在vRNIC中,虚拟机的前端位于guest内核空间,这是因为,在客户机内核实现,可以提供给verbs与原生一致的接口,从而无需修改虚拟机中的verbs用户库,减少了系统的复杂性。
% 容器的前端位于用户空间。因为容器前端位于主机内核空间,仍然无法实现对verbs库透明,因为此时主机内核中前端提供的设备接口可能与原生设备接口冲突。而前端实现在用户空间,verbs与前端之间的交互完全在容器应用之间的函数调用,更加自然和简单。唯一的代价是,对容器的verbs库进行少量的修改工作,以替换原verbs与内核驱动交互的接口。而且,替换在verbs用户库内部进行,不影响对上层应用程序的透明性。
The vRNIC frontend acts the interface of the vRNIC for the upper Verbs library in the guest environments. The serial Verbs commands can be forwarded into the frontend and be forwarded into vRNIC backends through frontend. For VMs, the frontend is in guest kernel space, and it simulates the native RDMA kernel driver to provided the same interface for upper Verbs library, such as device class and device file path. Thus, unmodified Verbs library can used in VMs when the vRNIC frontend is installed, and this make URN framework more stable. However, for containers, if the frontend is in host kernel space, there are still hard to make the interface transparent for upper Verbs library. Because the virtual interface may conflict with native RDMA device interface, and the Verbs library cannot distinguish them without extra modification. Thus, the frontend for containers is provided as a costumed library and linked to modified Verbs library. In specific, in Verbs library, system call about origin interface are replaced with function call into the frontend. Thus, another advantage is that the interact with upper Verbs library is wholly finished in user space without the overhead of system calls.  Finally, note that all frontends is hardware-independent because they only interact with the hardware-independent Verbs library and backends, and transparent to top applications.


% 后端:后端在哪,为什么用户空间?
% 在本文中,vRNIC 后端实现在主机的用户空间,而非主机内核空间,主要是基于下述三个方面的考虑:1,用户空间具有更高的管理灵活性:基于用户空间,URN可以敏捷迭代更多的管理功能;2,用户空间的稳定性更好:如果URN在内核空间实现,内核受到的攻击面更大,稳定性更差;3,用户空间的兼容性更好
vRNIC backend is locates in the host user space as a virtual device. There are three advantages for vRNIC backend in user space: first, the develop of management functions is easier and more flexible in user space; second, the attack surface is limited for user-space because of minimal inserted code into the kernel/hypervisor; third, the inserted codes are often dependent on specific APIs (e.g. system architecture, kernel version and device driver), thus, backend has higher compatibility in user space.


\subsection{Verbs and hardware-specific libraries}
% 1,为了实现上述vRNIC的完整功能,本文对于部分环境的用户库进行了修改:在容器中,对verbs库的修改主要是替换原生的与内核驱动交互的接口,该为与vRNIC前端交互,以便让verbs库的RDMA命令能够在用户空间转发到vRNIC前端。
According the design of vRNIC, there are some necessary modifications in Verbs library and hardware-specific library, e.g. libmlx4. The Verbs library is hardware-independent, which provides general interface to upper applications and interact with RDMA kernel driver through system call. The hardware-specific library mainly includes the key details of RDMA resources, such as allocate memory and fillwith QP. 

In URN, the modifications in Verbs library is mainly about interfaces because the interface may be changed; And the modifications in hardware-specific library is mainly memory mapping operations, and the resource memory address translation, because the memory mapping brings inconsistent ion between the applications' memory address and the RNIC's recorded registered address (in vRNIC backend address space).

Firstly, in host, we modified the memory allocation about RDMA resources in hardware-specific library. After modification, we can assign the mapped memory to register RDMA resources to physical RNIC.

For containers, we modified interface between Verbs library and RDMA kernel driver, and replace with the interface to vRNIC frontend. Note that mapping RDMA resource to backends can be finished frontends, because all RDMA resources are memory-aligned. For VMs, we do not modify the Verbs library due to the simulate interface is same as native RDMA. For both VMs and containers, the address translation is needed in data path.


% 此外,在host中对verbs及其设备相关的用户库进行了修改。因为,除了MR资源的地址由应用程序指定,RDMA 的QP等资源的内存分配等操作是封装在设备相关库完成的。在vRNIC后端中,为了实现数据路径对RDMA资源的映射,因此需要QP等RDMA资源的内存使用映射后的虚拟地址。

\subsection{URN Core}
There is one URN core instance in each host's user space. URN core is designed to manage the virtual and physical RDMA resources for the host, including instantiating vRNIC backends, managing virtual RDMA network configurations and policies. URN core also interact with physical RNIC through Verbs libraries. 
URN is designed to manage the virtual and physical RDMA resources for the host, including instantiating vRNIC backends, managing virtual RDMA network configurations and policies. Also, URN is in host user space for the same reasons in section xxx.

% URN core如何实例化vRNIC后端?
%对于使用vRNIC的虚拟机或容器, 需要构建与URN之间的消息通道,以能够发送vRNIC设备初始化请求,让URN core实例化vRNIC 的后端。对于虚拟机,URN core实现在用户空间, 而非在内核空间借助虚拟机监视器直接获取虚拟机的设备初始化请求。因此,URN和虚拟机之间需要建立额外的监听通道,以获取虚拟机的设备初始化请求。,本文采用Unix套接字,由URN core和对应虚拟机进程共享这一路径,虚拟机和容器在需要vRNIC时,便通过socket发出该vRNIC初始化请求, URN core 收到请求后便实例化对应的vRNIC后端。为了支持多个vRNIC后端的快速实例化,采用了reactor模式.
Recall that the vRNIC backends is in host user space, it is necessary to build a communication channel between URN and all guests, because the URN (host user space) are isolated with both containers and the hypervisor(VMs). When guests need a vRNIC, they can request the URN to instantiate corresponding backend in host user space. To build this channel, we utilize the UNIX socket for both VMs and containers. When VMs or containers need a new vRNIC, the request to socket are deal with URN. To support multiple request rapidly, the reactor mode are adopted in the process.

% URN core 对BE提供统一接口并统筹管理物理资源?
% URN要实现对虚拟RDMA网络的统一管理,需要监控后端的RDMA资源,并汇总资源映射关系。因此,UNR core 设计了对应的接口让后端在创建资源时调用,从而记录关键的资源信息;同时后端也提供了相应的接口给UNR core调用,从而可以实现对后端相应资源的控制。 ,例如,资源数目,与vRNIC映射关系,与RNIC映射关系等。通过这些实时记录和用户定义的管理策略,从而实现对虚拟机和容器使用RDMA资源的统一管控。例如, 在限制vRNIC的QP使用量后,如果某个vRNIC后端创建的QP资源过多,UNR core可以阻止该vRNIC后端后续的创建QP资源操作。 
Besides instantiating vRNIC backends, the URN core is also designed for the management of virtual RDMA network. Thus, the RDMA resources in vRNIC backends should be monitored and the resources mapping information (e.g. resource memory mapping) should be summarized in URN core. To achieve that, 
To achieve that, first, the URN provides the recording interfaces about RDMA resource information, and the interface is execute in vRNIC backends when RDMA resources are registered. second, the control interfaces are also provided and user-defined rules can be deployed into these interfaces. For example, if the maximum of QPs is defined for each vRNIC, when recorded QPs is excessive for a vRNIC, the URN core can execute the control operation, such as suspend the new QP request for the vRNIC.

\subsection{Discuss}
% 4.4 discussion 
% (1)云环境管理: 云环境中需要更多的管理功能,例如QoS,流量计费等策略。考虑到数据路径已经绕过了UNR Core,无法在UNR Core对RDMA的数据路径进行截获和管理,因此,本文将管理功能安插到对应的guest用户库中,例如QoS、ACL等。对于这些功能的管理策略,可以通过mgmt center进行统一分发到各个UNR core,再通过管理通道反馈到上层的guest lib库中。要实现这些管理功能, 需要guest用户对这些软件给予充分的信任,将包含进trusted computing base (TCB)。
(1) Management for clouds: There are lots of important managements in clouds, e.g. QoS, metering and so on. Apparently, these policy is mainly about the information about data path in RDMA, e.g. the message size. However, in URN, both vRNIC Frontend and Backend are bypassed in the data path. Thus, we should extend these management into specific-hardware library in guest, because data path are implemented in this library. For the policies, the management center can be used to arrange. Note that, these extension needs that the library is trusted for all user, and should be included into TCB (trusted computing base).


% (2)Guest迁移:容器或虚拟机的迁移在云环境中十分重要,例如,提供资源利用率和恢复故障等。URN可以很方便地支持offline的guest迁移,例如,将guest暂停然后打包迁移到远端服务器,然后重新启动便可以以原来的虚拟网络地址重新建立RDMA连接,此时只需要在URN core中更新虚拟网络地址与迁移后物理地址的对应关系,无需重新对物理网卡或交换机进行配置。对于live migrate, 由于RDMA在数据路径是绕过内核,同时单边操作远程是无感知的,因此,难以确定RDMA应用的内存页面状态,这一困难是RDMA虚拟化框架均面临的,需要借助其他研究工作进行解决。
(2) Virtual Instances Migration: Migration is important  for both containers and VMs in clouds with many benefits, e.g. resource utilization and fail-over. With the virtual RDMA network , URN can support offline migration without reconfiguring the physical RDMA network for applications. In specific,after rebooting the migrated virtual instance, the application can rebuild the RDMA connection  only by modifying the address mapping in software virtual layer. The physical RDMA device do not need to be reconfigured. Currently, for live migrations, it is still hard because memory regions in RDMA application may be uncertain under bypassing or one-side communication. Thus, URN does not support live migrations for both VMs and containers.

% (3) 其他网络扩展: RDMA也可以被用来加速其他网络应用,例如基于TCP/IP网络协议栈的socket应用。现有的工作包括vSocket和SockDirect等。在URN中,同样可以通过扩展已有架构来实现优化guest socket的效果。通过修改vRNIC的前后端,可以将guest的socket命令转发到后端由RDMA执行。对应的,还需要修改与容器中的socket用户库,以将命令直接通过前端转发到后端。在后端,将socket命令通过RDMA网络执行,然后返回给guest。对于数据路径,同样可以通过映射方式实现零拷贝。
(3) Other network extensions: RDMA can also be exploited to optimize the performance of other network applications,  such as TCP/IP.  Existing works include vSocket ~\cite{wang2019vsocket}, SockDirect~\cite{li2019socksdirect} so on. In URN, we can realize the similar functions through the .  
